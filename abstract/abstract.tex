\documentclass[a4paper]{article}

\usepackage[utf8]{inputenc}
\usepackage[english]{babel}
\usepackage{csquotes}
\usepackage{microtype}
\usepackage{authblk}

\usepackage[giveninits,maxbibnames=1000]{biblatex}
\bibliography{references}

\usepackage{geometry}
\geometry{a4paper,tmargin=25mm,bmargin=25mm,lmargin=30mm,rmargin=30mm}

\usepackage{baskervald}
\usepackage[T1]{fontenc}
\renewcommand*{\bibfont}{\small}

\title{Numerical recipes for environmental sciences with MATLAB}
\author[1]{Fabio Durastante}
\affil[1]{Università di Pisa, Dipartimento di Matematica}

\date{Ph.D. Course 20 hrs - 10 Lectures}

\begin{document}
	
\maketitle	
	
Numerical simulations are computations we run on a computer with programs implementing a mathematical model for a chemical, physical or biological system. We need them to study the behavior of processes whose mathematical formulations are too complex to provide analytical solutions. The computational science area is itself a rapidly growing field. While the largest and most accurate simulations often use advanced computing capabilities, there is an ample layer of small and intermediate problems across many disciplines that we can face with easier-to-handle tools. This course will address one of such tools called MATLAB. We will use it to perform small-scale computer simulations. In the first place, we are going to introduce the \emph{programming language} on its own and take some familiarity with it. Then, we will apply it to solve some problems in Earth, Life, and Chemical sciences. 

\paragraph{Scheduling.} Two lessons per week of two hours each from the 1\textsuperscript{st} of March. Tuesday and Thursday from 9.00 to 11.00.

\paragraph{Modalities.} The course will take place online in synchronous mode.

\nocite{*}
\printbibliography

\end{document}